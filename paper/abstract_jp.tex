ソフトウェアの変更漏れはバグ発生の主要な要因であり
バグの発生を防ぐため、開発者が次に変更するべき箇所を検出し提案する、変更支援と呼ばれる研究が行われている。
このうち、開発者の過去の開発行動が記録されている改版履歴を用いた研究では、改版履歴からロジカルカップリングと呼ばれる成果物間の関係を抽出して変更推薦に利用する手法が提案され、改版履歴を利用した変更支援に一定の効果があることが示された。
また、改版履歴を用いた手法よりも多くの変更推薦を行なうため、変更された成果物の情報だけでなく、開発者がプログラムの変更のために閲覧した成果物の情報を含んだ操作器歴を利用する研究がある。
操作履歴を用いる研究では共通して、改版履歴がほとんどの実用的なソフトウェア開発で利用されているのに対し、操作履歴を記録しているソフトウェア開発はほとんどなく、手法の効果の検証のためのデータセットが用意できないという問題がある。

本研究は、十分なデータセットを用いて操作履歴を用いた変更推薦手法の効果を検証することを目的とし、現存する最も普及しているツール、Mylynを利用して、Mylynの出力する操作履歴を変更支援手法に適用する方法について検討する。
Mylynの出力する操作履歴は既存手法にそのまま適用することができないので、2つの方法で変更推薦を実施する。
1つ目ではMylynを拡張し、Mylynの操作履歴に既存の変更支援手法に必要な情報を付け加えることで、既存の変更支援手法にMylynの操作履歴が適用できるようにする方法を提案する。
2つ目では、現バージョンのMylynの操作履歴を適用できるような変更支援手法を新たに提案し、既存の改版履歴を用いた手法と比較して手法の効果を検証した。

