\documentclass[a4paper]{jsbook}
%\documentclass[a4paper,oneside]{jsbook} % 片面刷りの場合は oneside を追加
%\documentclass[a4paper,openany]{jsbook} % 章末の白紙ページを省きたいとき

\usepackage[sort]{cite}              % 参考文献のソート

\usepackage[mthesis,draft]{salab}    % 草稿では draft を追加
%\usepackage[mthesis]{salab}         % 本番はこっち


\title{細粒度操作履歴の分析に基づく\\開発行動推薦}
\author{山森 章弘}
\authorintitle{山~~森~~~~~章~~弘} % 表示用
\date{平成28年1月}
\nendo{平成27}
\studentid{14M38439}
\advisor{小林 隆志 准教授}
\affiliation{大学院情報理工学研究科 計算工学専攻}

\begin{document}

\frontmatter
\maketitle

\chapter*{概要}
本論文では,...

\chapter*{Abstract}
This dissertation studies ...

\tableofcontents
\listoffigures
\listoftables

\mainmatter
% 章番号が決まったらファイル名に連番を振っても良いし,
% include せずにすべてをこのファイルに含めても良い.

\chapter{序論}
\section{背景}
ソフトウェアの大規模化に伴い、メンテナンス時のバグ修正や機能追加タスクの作業で必要な依存関係解決の作業量は増大している。
ソフトウェアの品質を向上させるため、メンテナンス作業中に変更の必要なソースコードを全て列挙できることは重要であり、この作業を支援する研究は変更支援(change guide)と呼ばれている。
変更支援のうち、静的解析に基づき、ソースコードの依存関係をもとに変更伝搬箇所を特定する手法が提案されている\cite{792645}が、
この手法は膨大な量の依存関係を解析するため計算量が多く、また全ての依存関係が変更の伝搬を引き起こすわけではないため効率が悪い\cite{Geipel:2009}。
また、静的解析では解決できないようなソフトウェアの依存関係に対しては変更支援を行なうことが出来ない\cite{5609732}。

静的解析による変更支援手法の限界を克服するために、
開発者の過去の開発履歴に注目して開発者に対して変更支援を行なう研究が行われている\cite{738508, Kagdi:2006}。
Zimmermannらは、改版履歴を解析することにより、変更される可能性のあるメソッド、フィールド等のコード要素を推薦する変更推薦ツール、eROSE\cite{Zimmermann:2005}を提案した。

最近では、開発者が過去に操作した統合開発環境(IDE)やWebブラウザ等の開発ツールの情報が記録された操作履歴(interaction history)\cite{rsse:2014}を解析する研究がさかんに行われている。
先行研究\cite{6233415,KatoJapanese:2011}では、変更推薦プロセスの性能を向上させるために、ソースコードへの変更と参照の情報を細粒に記録されている操作履歴を変更支援手法に適用した。
この研究では、開発者のコードへの参照行為を2つの変更行為の間のコンテクストとして利用し、変更推薦手法の精度を操作履歴によって向上させられる可能性を示した。
さらに、この手法を拡張し,変更間の時間的局所性を考慮した改善手法\cite{ss2012-76}や、
複数箇所に変更が波及する場合を考慮し推薦尤度の累積を行なう手法\cite{ss2013-84}が提案されている。

細粒度な操作履歴を用いた研究の共通の問題点として、
改版履歴とは異なり操作履歴には一般的に使われている記録ツールが存在せず、
手法の効果の一般性を実証的に示すことが困難なことが挙げられる。
操作履歴を収集するツールにはPLOG\cite{plog}、{\sc DFlow}\cite{minelli:2014}などがあるが、
これらのツールで記録できる操作履歴は取得できる操作の種類や粒度などが異なり、それぞれのツールで記録された操作履歴を相互に変換することはできない。
また、これらのツールの多くは研究目的で製作されたものであり、公開されているものは少ないため、現存する操作履歴は数日〜数週間程度の短いものがほとんどである。

Mylyn\cite{Kersten:2005}は公開されている操作履歴記録ツールであり、実開発で記録された操作履歴が7年分利用可能となっている。
しかし、Mylynで記録された操作履歴は他の細粒度操作履歴と異なり、操作の時系列を復元することが出来ないため、既存の操作履歴を利用した変更支援手法をそのまま適用することが出来ない。
したがって、Mylynで記録された操作履歴を適用できる変更支援手法を確立し変更支援を実行することで、
操作履歴を利用した変更支援が一般の開発作業でも適用可能であることを示すことが必要とされている。
\section{目的}
\section{本論文の構成}
\chapter{関連研究}
\section{変更波及解析による変更支援}
\section{改版履歴をマイニングする手法}
%\subsection{改版履歴を取得するツール}
\section{操作履歴をマイニングする手法}
\subsection{操作履歴を取得するツール}
\subsection{細粒度な操作履歴}
\subsection{粗粒度な操作履歴}
\chapter{Mylynの操作履歴}
\section{Kind}
\section{StructureHandle}
\section{Date}
\chapter{Mylynの操作履歴を細粒度にする拡張}
\section{概要}
\section{設計}
\section{実行例}
\chapter{提案手法}
\section{概要}
\section{操作履歴の取得}
\section{改版履歴と操作履歴の結合}
\section{相関ルールの抽出}
\section{評価メトリクス}
\chapter{評価実験}
%section分け未定
\chapter{結論}
\chapter{謝辞}
\appendix
%\chapter{付録A}
%\chapter{付録B}

\backmatter
\bibliographystyle{junsrt}
\bibliography{bibtex/citation}
% 最終稿段階で .bbl に細かい修正を入れること.

\end{document}
