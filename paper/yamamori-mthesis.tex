\documentclass[a4paper]{jsbook}
%\documentclass[a4paper,oneside]{jsbook} % 片面刷りの場合は oneside を追加
%\documentclass[a4paper,openany]{jsbook} % 章末の白紙ページを省きたいとき

\usepackage[sort]{cite}              % 参考文献のソート

\usepackage[mthesis,draft]{salab}    % 草稿では draft を追加
%\usepackage[mthesis]{salab}         % 本番はこっち


\title{細粒度操作履歴の分析に基づく\\開発行動推薦}
\author{山森 章弘}
\authorintitle{山~~森~~~~~章~~弘} % 表示用
\date{平成28年1月}
\nendo{平成27}
\studentid{14M38439}
\advisor{小林 隆志 准教授}
\affiliation{大学院情報理工学研究科 計算工学専攻}

\begin{document}

\frontmatter
\maketitle

\chapter*{概要}
本論文では,...

\chapter*{Abstract}
This dissertation studies ...

\tableofcontents
\listoffigures
\listoftables

\mainmatter
% 章番号が決まったらファイル名に連番を振っても良いし,
% include せずにすべてをこのファイルに含めても良い.

\chapter{序論}
\section{背景}
\section{目的}
\section{本論文の構成}
\chapter{関連研究}
\section{変更波及解析による変更支援}
\section{改版履歴をマイニングする手法}
%\subsection{改版履歴を取得するツール}
\section{操作履歴をマイニングする手法}
\subsection{操作履歴を取得するツール}
\subsection{細粒度な操作履歴}
\subsection{粗粒度な操作履歴}
\chapter{Mylynの操作履歴}
\section{Kind}
\section{StructureHandle}
\section{Date}
\chapter{提案手法}
\section{概要}
\section{操作履歴の取得}
\section{改版履歴と操作履歴の結合}
\section{相関ルールの抽出}
\section{評価メトリクス}
\chapter{評価実験}
%section分け未定
\chapter{結論}
\chapter{謝辞}
\appendix
%\chapter{付録A}
%\chapter{付録B}

\backmatter
\bibliographystyle{IEEEtran}
\bibliography{bibtex/citation}
% 最終稿段階で .bbl に細かい修正を入れること.

\end{document}
