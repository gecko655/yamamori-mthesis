\documentclass[a4paper]{jsbook}
%\documentclass[a4paper,oneside]{jsbook} % 片面刷りの場合は oneside を追加
%\documentclass[a4paper,openany]{jsbook} % 章末の白紙ページを省きたいとき

\usepackage[sort]{cite}              % 参考文献のソート

\usepackage[mthesis,draft]{salab}    % 草稿では draft を追加
%\usepackage[mthesis]{salab}         % 本番はこっち

\hyphenpenalty=10000

\title{細粒度操作履歴の分析に基づく\\開発行動推薦}
\author{山森 章弘}
\authorintitle{山~~森~~~~~章~~弘} % 表示用
\date{平成28年1月}
\nendo{平成27}
\studentid{14M38439}
\advisor{小林 隆志 准教授}
\affiliation{大学院情報理工学研究科 計算工学専攻}

\begin{document}

\frontmatter
\maketitle

\chapter*{概要}
本論文では,...

\chapter*{Abstract}
This dissertation studies ...

\tableofcontents
\listoffigures
\listoftables

\mainmatter
% 章番号が決まったらファイル名に連番を振っても良いし,
% include せずにすべてをこのファイルに含めても良い.

\chapter{序論}
\section{背景}
ソフトウェアの大規模化に伴い、メンテナンス時のバグ修正や機能追加タスクの作業で必要な依存関係解決の作業量は増大している。
ソフトウェアの品質を向上させるため、メンテナンス作業中に変更の必要なソースコードを全て列挙できることは重要であり、この作業を支援する研究は変更支援(change guide)と呼ばれている。
変更支援のうち、静的解析に基づき、ソースコードの依存関係をもとに変更伝搬箇所を特定する手法が提案されている\cite{792645}が、
この手法は膨大な量の依存関係を解析するため計算量が多く、また全ての依存関係が変更の伝搬を引き起こすわけではないため効率が悪い\cite{Geipel:2009}。
また、静的解析では解決できないようなソフトウェアの依存関係に対しては変更支援を行なうことが出来ない\cite{5609732}。

静的解析による変更支援手法の限界を克服するために、
開発者の過去の開発履歴に注目して開発者に対して変更支援を行なう研究が行われている\cite{738508, Kagdi:2006}。
Zimmermannらは、改版履歴を解析することにより、変更される可能性のあるメソッド、フィールド等のコード要素を推薦する変更推薦ツール、eROSE\cite{Zimmermann:2005}を提案した。

最近では、開発者が過去に操作した統合開発環境(IDE)やWebブラウザ等の開発ツールの情報が記録された操作履歴(interaction history)\cite{rsse:2014}を解析する研究がさかんに行われている。
先行研究\cite{6233415,KatoJapanese:2011,ss2012-76,ss2013-84,Yamamori:2016}では、変更推薦プロセスの性能を向上させるために、ソースコードへの変更と参照の情報を細粒に記録されている操作履歴を変更支援手法に適用した。

細粒度な操作履歴を用いた研究の共通の問題点として、
改版履歴とは異なり操作履歴には一般的に使われている記録ツールが存在せず、
手法の効果の一般性を実証的に示すことが困難なことが挙げられる。
操作履歴を収集するツールにはPLOG\cite{plog}、{\sc DFlow}\cite{minelli:2014}などがあるが、
これらのツールで記録できる操作履歴は取得できる操作の種類や粒度などが異なり、それぞれのツールで記録された操作履歴を相互に変換することはできない。
また、これらのツールの多くは研究目的で製作されたものであり、公開されているものは少ないため、現存する操作履歴は数日〜数週間程度の短いものがほとんどである。

Mylyn\cite{Kersten:2005}は公開されている操作履歴記録ツールであり、実開発で記録された操作履歴が7年分利用可能となっている。
しかし、Mylynで記録された操作履歴は他の細粒度操作履歴と異なり、操作の時系列を復元することが出来ないため、既存の操作履歴を利用した変更支援手法をそのまま適用することが出来ない。
したがって、Mylynで記録された操作履歴を適用できる変更支援手法を確立し変更支援を実行することで、
操作履歴を利用した変更支援が一般の開発作業でも適用可能であることを示すことが必要とされている。
\section{目的}
\section{本論文の構成}
\chapter{関連研究}
\section{変更波及解析による変更支援}
ソースコードの静的解析を用いた変更伝搬箇所特定の研究が行われてきた\cite{792645}。
このような研究では、ソースコードを解析することによって、メソッドコールやクラス継承、フィールド参照等の依存関係を取得している。
しかし、ソフトウェアの内部にはそのような依存関係が無数に存在し、その全てが変更伝搬に関係しているわけではない。
Geipelら\cite{Geipel:2009}は、半分以上のクラス同士の依存関係は変更伝搬に全く関係ない上、ごく一部の依存関係が殆どの変更伝搬に関係していると報告した。
さらに、Canforaら\cite{5609732}は静的解析では一部の依存関係を見つけられないことを報告した。
以上のように、ソースコードの依存関係のみを用いた変更支援は困難であるといわれてきた。

\section{改版履歴をマイニングする手法}
静的解析による変更支援手法の限界を克服するため、ソフトウェア成果物の変更の歴史を利用し、メンテナンス時に必要な変更箇所を推薦するような研究が進められている。

Gallら\cite{738508}はCVSのようなバージョン管理システムに保存された改版履歴を解析し、logical couplingと呼ばれる同時に変更されやすいファイル同士に張られる関係を抽出する手法を提案した。

ZimmermannらはeROSEというツール\cite{Zimmermann:2005}を実装した。
このツールはメソッド粒度でlogical couplingを抽出し、開発者があるメソッドを変更した時に外の変更するべきメソッドを推薦する機能を持つ。
Kagdiらはsqminerというツール\cite{Kagdi:2006}を実装した。
このツールは改版履歴に含まれる変更セットから、logical couplingでは考慮していなかった変更の時系列を擬似的に計算し、変更支援の精度を向上させることに成功した。
Gerardら\cite{5609732}は、グレンジャー因果性検定を使ってバージョン管理システムの改版履歴から相関ルールを抽出し、変更予測を補完する研究を行なった。

これらのCVSやSubversion、Gitなどのようなバージョン管理システムの改版履歴を利用した変更支援の研究は2つの限界が存在する。
1つ目は、バージョン管理システムの改版履歴に保存されるソフトウェア成果物の変更の情報はコミット時のみに記録されるということである。
したがって、変更の発生した正確な時間が改版履歴には記録されない。
2つ目は、改版履歴には変更情報しか記録されず、参照情報が記録されないことである。
参照されたソフトウェア成果物の情報は変更情報と同様に変更支援の情報源となる可能性がある。
\section{操作履歴をマイニングする手法}
改版履歴に代えて、改版履歴よりも細かい粒度で開発者の活動履歴が記録されている操作履歴を保存し、マイニングする手法が提案された\cite{Hill:1992}。
ここで用いられている操作履歴(interaction history)とは、改版履歴でも記録できるファイルの変更の記録だけでなく、選択や参照などの変更を伴わない開発者の操作や、それらの操作の時系列が記録されたものを指す。

本論文では、操作履歴のうち「変更と操作の区別」と操作の正確な時系列が記録されているものを「細粒度操作履歴」と呼び、それ以外を「粗粒度操作履歴」と呼ぶ。
\subsection{細粒度操作履歴をマイニングする手法}
Zouら\cite{4268248}は、メンテナンス作業中のファイルの特徴として、interaction couplingを定義した。
interaction couplingとは、頻繁に同時参照されているファイル間に張られる関係である。
Robbesら\cite{Robbes:2008}も、細粒度操作履歴におけるlogical couplingとして同様の概念を提案した。
彼らは様々な変更予測手法を操作履歴に対して適用し、最近変更されたファイルに基づく手法が最も正確であることを示した\cite{5463278}。

様々な目的で細粒度操作履歴を用いる手法が広く研究されている。
Maalejら\cite{Maalej:2010}は、複数の開発ツールで記録した操作履歴を用いて、
開発者が次に使うべき開発ツールを推薦する手法を提案した。
Roehmらは、コードの変更、Webサーチ、コンパイルエラー等の履歴を収集し、隠れマルコフモデルを用いて開発者が問題解決する段階を表現する過程を可視化した。

小林ら\cite{6233415,KatoJapanese:2011}は、変更と参照の情報が記録された操作履歴を学習して変更支援グラフを作成し、このグラフを基に変更推薦をする手法を提案した。
この研究では、開発者のコードへの参照行為を2つの変更行為の間のコンテクストとして利用し、変更推薦手法の精度を操作履歴によって向上させられることを示した。
さらに、この手法を拡張し,変更間の時間的局所性を考慮した改善手法\cite{ss2012-76}や、
複数箇所に変更が波及する場合を考慮し推薦尤度の累積を行なう手法\cite{ss2013-84,Yamamori:2016}が提案されており、
\cite{Yamamori:2016}では、メソッド粒度の変更推薦において精度が向上することを示した。

細粒度操作履歴を用いて成功している手法の多くは、実証実験を行なうためのデータが不足していることが共通の問題となっている。
これは細粒度操作履歴を記録するためのツールが一般に普及しておらず、多くの研究で実証実験のために記録された短期間の操作履歴を用いているためである。
表\ref{finegrained}は細粒度操作履歴を収集し出力できるツールをまとめたものである。
これらの5つのツールは、開発者の操作を、正確な時間や時系列、変更したかどうかの情報を含めて記録することができる。
この3つの情報はMylynでは記録することができない。

\begin{table}[bt]
  \caption{操作履歴を収集し出力できるツールの一覧}
  \centering
  \begin{tabular}{ll}
    \hline
    ツール名& 操作の種類\\
    \hline
    PLOG\cite{plog} & ファイルへのアクセス、変更の有無\\
           & カーソル行のメソッド名、 標準出力やエラー出力の内容 \\
    FLUORITE\cite{yoon:2011} & ファイルへの挿入や削除、
          コードの行数、 実行等  \\
    CodingTracker & 
    テキストの編集、エディタの比較、 リファクタリング、\\
    \cite{Negara:2012}\cite{Negara:2014}  & 
    バージョン管理システムの操作, 
    JUnitテストの実行、 起動等\\
    {\sc DFlow}\cite{minelli:2014} & コードの読み書き、 ソフトウェア検査等\\
    IDE++\cite{Gu:2014} & キーストローク(キーの名前)、 実行、 \\
                        & リファクタリング、 保存等\\
    \hline
  \end{tabular}
\label{finegrained}
\end{table}

PLOG\cite{plog}は開発者のコードへの参照と編集を、時間と時系列とともに記録するツールである。
また、実行時の実行時例外の発生も記録できる。
YoonらはFLUORITE\cite{yoon:2011}を開発した。
このツールは、ファイルへの挿入や削除、コードの行数や抽象構文木中に含まれるノードの数などを記録でき、また、これらの情報の変化を可視化する機能がある。
この可視化情報をもとに、開発者はソフトウェアの成長過程を見ることができる。
NegaraらはCodingTracker\cite{Negara:2012}という操作履歴記録ツールを実装した。
このツールは、Eclipseのリファクタリング機能の操作などといった38種類のコード進化イベントを記録できる。
彼らは、CodingTrackerを用いて記録された操作履歴を分析し、幾つかのコード変更パターンを見つけ出した\cite{Negara:2014}。
Minelliらは{\sc DFlow}というツールを開発した。
このツールは、PharoというSmalltalk開発用統合開発環境用に実装された操作履歴記録ツールで、Smalltalkの特徴である検査操作などを含む33種類の詳細な操作イベントを記録できる。
彼らは、{\sc DFlow}とPLOGを用いてそれぞれ記録された操作履歴を解析し、開発者のコード理解の段階を明らかにした。
Guらは、キーストロークのような最も細かい粒度で44種類の操作イベントを記録できるツールIDE++\cite{Gu:2014}を開発した。

これらのツールはすべて研究段階にあり、実開発の現場で利用され操作履歴が記録されたものはほとんどない。
したがって、現時点で記録された操作履歴には1年以上の期間記録されたものはないため、数年分の細粒度操作履歴を用いて手法の実証実験を行なうことは非常に困難である。
これは、多くのオープンソースソフトウェアの開発で数年分の改版履歴が記録されており、これを利用して改版履歴を利用した研究が盛んに行われている事実とは対照的である。
\subsection{粗粒度操作履歴をマイニングする手法}
Kerstenらは、開発者が頻繁に操作しているソフトウェア成果物群を検知することで、
現在の開発者が必要としている成果物を特定し集約する手法を提案した。
彼らは、統合開発環境上で集約された成果物を開発者に表示するEclipseプラグイン、``Mylar"(現在の``Mylyn")\cite{Kersten:2005}を実装した。
Mylynは開発者の粗粒度操作履歴を自動で記録する機能が付属している。
Mylynの操作履歴はMylyn自身の開発プロジェクトで2007年から記録が開始され、{\it Bugzilla}上からダウンロード可能となっている。
したがって、Mylynで記録されたこの粗粒度操作履歴を利用して実証実験を行なうことで、前節で挙げた細粒度操作履歴用いる研究の問題を解消することができる。
しかしながら、この粗粒度操作履歴は多くの細粒度操作履歴とは異なり、時間の情報が欠落しており、変更と参照の区別がない(\ref{mylyn_chap}章を参照)。
細粒度操作履歴を用いた手法に粗粒度操作履歴をそのまま適用させることは出来ないため、粗粒度操作履歴を利用できる新たな手法を生み出す必要がある。

T. Leeら\cite{TLee:2011}はMylynの粗粒度操作履歴を用いた最初の研究を行なった。
彼らは操作履歴をmicro interaction metrics (MIMs)を計算するために利用し、MIMsによって欠陥予測の精度を向上させた。
Blincoeら\cite{Blincoe:2012}はMylynの粗粒度操作履歴を用いて開発者とタスクとの近接度({\it proximity})を求めた。
Robbesら\cite{Robbes:2013}とRicardら\cite{Silva:2015}らも同様にMylynの粗粒度操作履歴から開発者の専門度メトリクスを求めた。
\cite{Blincoe:2012,Robbes:2013,Silva:2015}によってMylynの操作履歴は開発者をタスクに振り分ける際に有効であることが示された。

Zanjaniら\cite{Zanjani:2014}は、変更リクエスト上でテキストマイニング手法を適用し、
Mylynの粗粒度操作履歴のバグ説明文と改版履歴のコミットメッセージを突き合わせることで機能捜索(feature location)を行った。
彼らは、改版履歴とMylynの粗粒度操作履歴の両方を使うことで機能捜索を効率的に行えることを示した。
Zanjaniらの手法はコミットや操作と関係する単語をクエリとして用意する必要があり、変更やコミットをクエリとして変更推薦を行なう本論文の手法とは異なる。

Bantelayら\cite{Bantelay:2013}は、interaction couplingを改版履歴とMylynの粗粒度操作履歴の両方から計算し、両方の履歴を利用することで、それぞれの履歴のみを利用するよりも再現率が向上することを示した。
しかし、この研究には2つの問題点がある。
まず、この研究ではMylynが記録する``selection",``edit",``propagation"などのような操作イベントの種類(\ref{kind_sec}節参照)を考慮せずに全て同一に利用している。
開発者の操作を解析する上では``edit"のイベントのみを利用するべきである。
次に、この研究では、バグのIDのみを用いてMylynの操作履歴とコミットを紐付けている。
Bugzillaにアップロードされた操作履歴には、開発中に記録された操作履歴の他にクラッシュリポートとしてコミットとは全く関係なくアップロードされた操作履歴を含むため、より慎重に操作履歴とコミットを紐付けるべきである。

S. Leeら\cite{SLee:2015}はMylynの粗粒度操作履歴から相関ルールを抽出する推薦システム、{\it MI}を、eROSE\cite{Zimmermann:2005}を拡張することで開発した。
彼らはMylynのselectionイベントをコードへの参照として、editイベントをコードへの変更としてそれぞれ利用した。
しかし、実際にはeditイベントは開発者がコードを見るだけで発生し、コードが変更されたかどうかとは関係がないため、この点について正しく解釈した研究が必要とされている。

\chapter{Mylynの操作履歴}\label{mylyn_chap}
Mylyn\cite{Kersten:2005}はタスクマネージメントツールであり、Eclipseプラグインとして実装されている。
Mylynの主な機能はタスクに関係するファイルを集約することだが、この機能に付随して、タスクが有効化されている間、開いたファイルの情報やカーソルの移動の情報等を操作イベントとして記録し、XML形式のログファイルに出力する機能がある。
本稿では、このログファイルのことを``Mylynログ"と呼び、あるタスクを実行中に記録された操作の列を``セッション"、セッションの列を``(Mylynの)操作履歴"と呼ぶ。
また、それぞれのセッションに含まれる1つの操作を``操作イベント"と呼ぶ。

MylynプロジェクトはイシュートラッキングシステムであるBugzilla\footnote{\url{https://bugs.eclipse.org/bugs/}}を利用している。
MylynプロジェクトのすべてのコミッターはMylynを用いてBugzillaにMylynログをアップロードする義務を負っているため\footnote{\url{https://wiki.eclipse.org/Mylyn/Contributor_Reference\#Using_Mylyn}}、Mylynログは2007年からBugzilla上にアップロードされ続け、現存する最も長期間の操作履歴となっている。

操作イベントはXMLの要素で表され、{\it Kind}、{\it StructureHandle}などの属性が存在する。
以下の節ではそれぞれの属性について詳しく説明する。

\section{Kind}\label{kind_sec}
\section{StructureHandle}
\section{Date}
\section{Interest}
\section{NumEvent}
\chapter{Mylynの操作履歴を細粒度にする拡張}
\section{概要}
\section{設計}
\section{実行例}
\chapter{Mylynの操作履歴による変更推薦手法}
\section{概要}
\section{操作履歴の取得}
\section{改版履歴と操作履歴の結合}
\section{相関ルールの抽出}
\section{評価メトリクス}
\chapter{評価実験}
%section分け未定
\chapter{結論}
\chapter{謝辞}
\appendix
%\chapter{付録A}
%\chapter{付録B}

\backmatter
\bibliographystyle{junsrt}
\bibliography{bibtex/citation}
% 最終稿段階で .bbl に細かい修正を入れること.

\end{document}
