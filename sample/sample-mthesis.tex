\documentclass[a4paper]{jsbook}
%\documentclass[a4paper,oneside]{jsbook} % 片面刷りの場合は oneside を追加
%\documentclass[a4paper,openany]{jsbook} % 章末の白紙ページを省きたいとき

\usepackage[sort]{cite}              % 参考文献のソート

\usepackage[mthesis,draft]{salab}    % 草稿では draft を追加
%\usepackage[mthesis]{salab}         % 本番はこっち


\title{ソフトウェア工学の研究}
\author{名字 名前}
\authorintitle{名~~字~~~~~名~~前} % 表示用
\date{平成21年1月}
\nendo{平成20}
\studentid{0xMxxxxx}
\advisor{佐伯 元司 教授}
\affiliation{大学院情報理工学研究科 計算工学専攻}

\begin{document}

\frontmatter
\maketitle
% -*- latex -*-

\chapter*{概要}
本論文では,...

\chapter*{Abstract}
This dissertation studies ...

\tableofcontents
\listoffigures
\listoftables

\mainmatter
% 章番号が決まったらファイル名に連番を振っても良いし,
% include せずにすべてをこのファイルに含めても良い.
% -*- latex -*-

\chapter{はじめに}
\label{c:introduction}

本章では,このスタイルの使い方を述べる.

\section{トピック・センテンス}

\TS{段落を代表する文をトピック・センテンスと呼ぶ.}
トピック・センテンスは段落に唯一あること,段落の最初の文であることが望ましい.
トピック・センテンスを適切に記述することにより,段落が主張したいことが読者に伝わりやすくなり,論文全体の可読性が向上する.

\TS{本スタイルではトピック・センテンスを{\tt $\backslash$TS\{...\}}でマークアップする.}
マークアップされたトピック・センテンスは,draftモードでは太字で,finalモード(draftではないモード)では特に装飾されることなく表示される.
原稿を先輩に見て貰うときは,draftモードで出力し,トピックが適切であるかをチェックして貰うこと.
また,このようにマークアップすることで,トピック・センテンスを意識して段落を記述することも習慣づけられる.
最終的な出力の際にはfinalモードとすれば綺麗な出力が得られ,原稿を最終版とする上でマークアップを外す必要はない.
なお,verbコマンドなど,一部のコマンドはTSの中に入れられないので,解決しがたいエラーが出たときはおとなしくTSで括るのを諦めよう.

\section{TODO}

\TS{本スタイルではTODO記述を{\tt $\backslash$TODO\{...\}}でマークアップする.}
マークアップされたTODOは,draftモードでは太字で強調されて出力されるが,finalモードでは一切出力されない.
先輩に観て貰うときには,直したいと思っている方向性などをこれで書いておくとよい.

\section{本論文の構成}
\TS{本論文の構成を以下に示す.}
本論文は全\ref{c:conclusion}章から成る.
2章以降の概要は以下の通りである.

% {\bf 「第\ref{c:xxx}章 \titleref{c:xxx}」では ... 

最後に,{\bf 「第\ref{c:conclusion}章 \titleref{c:conclusion}」}で本論文をまとめる.

%\include{background}
%\include{relatedwork}
%\include{approach}
%\include{tool}
%\include{casestudy}
%\include{discussion}
% -*- latex -*-

\chapter{おわりに}
\label{c:conclusion}


\appendix
%\chapter{付録A}
%\chapter{付録B}

\backmatter
% -*- latex -*-

\chapter{謝辞}
\label{c:acknowledgement}

謝辞の文章は定型文じゃないので載せません.自分で過去の論文を参照したり文章を考えたりしてください.

\bibliographystyle{junsrt}
\bibliography{bibtex/references}
% 最終稿段階で .bbl に細かい修正を入れること.

\end{document}
