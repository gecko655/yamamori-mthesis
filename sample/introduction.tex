% -*- latex -*-

\chapter{はじめに}
\label{c:introduction}

本章では,このスタイルの使い方を述べる.

\section{トピック・センテンス}

\TS{段落を代表する文をトピック・センテンスと呼ぶ.}
トピック・センテンスは段落に唯一あること,段落の最初の文であることが望ましい.
トピック・センテンスを適切に記述することにより,段落が主張したいことが読者に伝わりやすくなり,論文全体の可読性が向上する.

\TS{本スタイルではトピック・センテンスを{\tt $\backslash$TS\{...\}}でマークアップする.}
マークアップされたトピック・センテンスは,draftモードでは太字で,finalモード(draftではないモード)では特に装飾されることなく表示される.
原稿を先輩に見て貰うときは,draftモードで出力し,トピックが適切であるかをチェックして貰うこと.
また,このようにマークアップすることで,トピック・センテンスを意識して段落を記述することも習慣づけられる.
最終的な出力の際にはfinalモードとすれば綺麗な出力が得られ,原稿を最終版とする上でマークアップを外す必要はない.
なお,verbコマンドなど,一部のコマンドはTSの中に入れられないので,解決しがたいエラーが出たときはおとなしくTSで括るのを諦めよう.

\section{TODO}

\TS{本スタイルではTODO記述を{\tt $\backslash$TODO\{...\}}でマークアップする.}
マークアップされたTODOは,draftモードでは太字で強調されて出力されるが,finalモードでは一切出力されない.
先輩に観て貰うときには,直したいと思っている方向性などをこれで書いておくとよい.

\section{本論文の構成}
\TS{本論文の構成を以下に示す.}
本論文は全\ref{c:conclusion}章から成る.
2章以降の概要は以下の通りである.

% {\bf 「第\ref{c:xxx}章 \titleref{c:xxx}」では ... 

最後に,{\bf 「第\ref{c:conclusion}章 \titleref{c:conclusion}」}で本論文をまとめる.
